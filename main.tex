\documentclass[11pt, a4paper]{scrartcl}
\usepackage{xcolor}
\usepackage[colorlinks=true,allcolors=blue,pdfborder={0 0 0},pdfstartview={FitV},breaklinks,linktocpage]{hyperref}
\usepackage{color,amsmath,amsfonts,amssymb}
\usepackage[utf8]{inputenc}
\usepackage{enumitem}%für mehrspaltige Aufzählungen
\usepackage[ngerman, english]{babel}
\usepackage{epsfig}
\usepackage{pstricks}
\usepackage{graphics}
\graphicspath{{Figures/}}
\usepackage{bbm}
%\usepackage{showlabels}
\usepackage{siunitx}
\usepackage[labelfont=bf]{caption}
%\usepackage{nonfloat}
\usepackage{booktabs}%für Tabellen
\usepackage{pgffor} %for loops
\usepackage{ifthen} %for if
\usepackage{pgfplots} %für Plots
\pgfplotsset{
compat=1.5,
every axis/.append style={line width=1.0pt, tick style={line width=1.0pt}}
} %für richtigen Abstand der labels
\usetikzlibrary{shapes,arrows,fit,calc,positioning,arrows}
\usepackage[
  textheight=25cm,
left=2.3cm,
right=2.3cm,
headheight=25pt,
  includehead,includefoot,
  heightrounded,
]{geometry}
\makeatletter
\def\input@path{{./Figures/}}
\usepackage{fancyhdr}
\pagestyle{fancy}
\fancyfoot{}
\fancyhead{}
\fancyhead[L]{}
\fancyhead[R]{\thepage}
\fancyhead[C]{\nouppercase{\leftmark}}
\usepackage{url}
\usepackage{amsthm}
\usepackage{graphicx}
\usepackage{tikzpagenodes}

\definecolor{darkgreen}{rgb}{0.0,0.6,0.0}
\definecolor{shadedgreen}{rgb}{0.8,0.95,0.8}
\definecolor{shadedblue}{rgb}{0.85,0.85,1.0}
\definecolor{darkblue}{rgb}{0.0,0.0,0.5}


\newlength\figureheight 
\newlength\figurewidth


\usepackage[varg]{txfonts} %font/Schrift





\definecolor{darkgreen}{rgb}{0.0,0.5,0.0}

% Aliases
\input{aliases.tex}
\input{bandstructure_SCF_commands.tex}
\newcommand{\XMAXBSSCFSOC}{1.3434623097571403}
\newcommand{\YMINBSSCFSOC}{-3.5}
\newcommand{\YMAXBSSCFSOC}{5.24582033}
\newcommand{\PLOTSBSSCFSOC}{
\addplot[very thick, darkblue, smooth] table {data/band_SCF_SOC14.dat};
\addplot[very thick, orange, dashed, smooth] table {data/band_SCF_SOC15.dat};
\addplot[very thick, darkblue, smooth] table {data/band_SCF_SOC16.dat};
\addplot[very thick, orange, dashed, smooth] table {data/band_SCF_SOC17.dat};
\addplot[very thick, darkblue, smooth] table {data/band_SCF_SOC18.dat};
\addplot[very thick, orange, dashed, smooth] table {data/band_SCF_SOC19.dat};
\addplot[very thick, darkblue, smooth] table {data/band_SCF_SOC20.dat};
\addplot[very thick, orange, dashed, smooth] table {data/band_SCF_SOC21.dat};
\addplot[very thick, darkblue, smooth] table {data/band_SCF_SOC22.dat};
\addplot[very thick, orange, dashed, smooth] table {data/band_SCF_SOC23.dat};
\addplot[very thick, darkblue, smooth] table {data/band_SCF_SOC24.dat};
\addplot[very thick, orange, dashed, smooth] table {data/band_SCF_SOC25.dat};
\addplot[very thick, darkblue, smooth] table {data/band_SCF_SOC26.dat};
\addplot[very thick, orange, dashed, smooth] table {data/band_SCF_SOC27.dat};
\addplot[very thick, darkblue, smooth] table {data/band_SCF_SOC28.dat};
\addplot[very thick, orange, dashed, smooth] table {data/band_SCF_SOC29.dat};
\addplot[very thick, darkblue, smooth] table {data/band_SCF_SOC30.dat};
\addplot[very thick, orange, dashed, smooth] table {data/band_SCF_SOC31.dat};
\addplot[very thick, darkblue, smooth] table {data/band_SCF_SOC32.dat};
\addplot[very thick, orange, dashed, smooth] table {data/band_SCF_SOC33.dat};
}
\newcommand{\XTICKLABELSSCFSOC}{xticklabels={$\Gamma$, M, K, $\Gamma$},}
\newcommand{\TICKSSPECIALKPOINTSBSSCFSOC}{0.0, 0.5, 0.8725291934868997, 1.3434623097571405}

\input{bandstructure_G0W0_commands.tex}
\input{bandstructure_G0W0_SOC_commands.tex}
\newcommand{\PLOTSDOSSCF}{
\addplot[thick, darkblue] table[x index=1,y index=0] {data/DOS_SCF.dat};
\addplot[thick] table[x index=1,y index=0] {data/PDOS_SCF_Mo.dat};
\addplot[thick] table[x index=1,y index=0] {data/PDOS_SCF_S.dat};
}

\newcommand{\PLOTSDOSG0W0}{
\addplot[thick, darkblue, smooth] table[x index=1,y index=0] {data/DOS_G0W0.dat};
\addlegendentry{DOS}\addplot[thick, smooth] table[x index=1,y index=0] {data/PDOS_G0W0_Mo.dat};
\addlegendentry{PDOS Mo}\addplot[thick, smooth] table[x index=1,y index=0] {data/PDOS_G0W0_S.dat};
\addlegendentry{PDOS S}}


\begin{document}

\pagenumbering{Roman}

\begin{titlepage}\pdfbookmark[0]{Title}{Title}
  \sffamily
  \begin{center}
{
\includegraphics[width=5cm]{cp2k_logo.png}
\\[1em]
\Huge \bfseries Output summary of a \GW calculation\\[0.4em] using the CP2K package}
  \end{center}{\large
  \vspace{3em}
  Contact:
  \\[1em]
    Jan Wilhelm
      \\[0.5em]
    Institute of Theoretical Physics
    \\[0.5em]
    University of Regensburg
    \\[0.5em]
    jan.wilhelm@ur.de
    \\[3em]
    
\pdfbookmark[0]{Contents}{Contents}
\tableofcontents
  }
\end{titlepage}







\pagenumbering{arabic}
\pagestyle{plain}






\pagestyle{fancy}

\section{Input parameters}


\section{Band structure}
\begin{figure}[h!]
\centering
\setlength\figureheight{11cm} 
\setlength\figurewidth{0.5\textwidth}
\begin{tikzpicture}
\begin{axis}[
height=\figureheight,
width=\figurewidth,
tick align=outside,
tick pos=left,
x grid style={white!70!black},
xmajorgrids,
xmin=0, 
xmax=\XMAXBSSCF,
xtick style={color=black},
xtick={\TICKSSPECIALKPOINTSBSSCF},
\XTICKLABELS
minor y tick num=1,
grid=both,
y grid style={white!70!black},
ylabel={$E-E_\text{VBM}$ in eV},
ymajorgrids,
ymin=\YMINBSSCF,
ymax=\YMAXBSSCF,
ytick style={color=black}
]
\PLOTSBSSCF
\end{axis}
\end{tikzpicture}

\hspace{-2.5em}
\setlength\figurewidth{0.4\textwidth}
\begin{tikzpicture}
\begin{axis}[
height=\figureheight,
width=\figurewidth,
tick align=outside,
tick pos=left,
x grid style={white!70!black},
xmajorgrids,
xlabel={Density of states (1/eV)},
%\XTICKLABELSSCF
minor y tick num=1,
grid=both,
y grid style={white!70!black},
ylabel={},
ymin=\YMINBSSCF,
ymax=\YMAXBSSCF,
ymajorgrids,
yticklabels={},
ytick style={draw=none},
legend pos=north west,
]
\addlegendimage{empty legend}
\addlegendentry{SCF}
\PLOTSDOSSCF
\end{axis}
\end{tikzpicture}

\caption{Electronic band structure computed from the self-consistent field (SCF) electronic structure method (left) and computed from $G_0W_0$ (right). The energy of the bands is given relative to the valence band maximum (VBM).}
    \label{f2}
\end{figure}



\begin{figure}[h!]
\centering
\setlength\figureheight{11cm} 
\setlength\figurewidth{0.47\textwidth}
\begin{tikzpicture}
\begin{axis}[
height=\figureheight,
width=\figurewidth,
tick align=outside,
tick pos=left,
x grid style={white!70!black},
xmajorgrids,
xmin=0, 
xmax=\XMAXBSGW,
xtick style={color=black},
xtick={\TICKSSPECIALKPOINTSBSGW},
\XTICKLABELSGW
minor y tick num=1,
grid=both,
y grid style={white!70!black},
ylabel={$E-E_\text{VBM}$ in eV},
ymajorgrids,
ymin=\YMINBSGW,
ymax=\YMAXBSGW,
ytick style={color=black}
]
\PLOTSBSGW
\end{axis}
\end{tikzpicture}

\hspace{-2.5em}
\setlength\figurewidth{0.4\textwidth}
\begin{tikzpicture}
\begin{axis}[
height=\figureheight,
width=\figurewidth,
tick align=outside,
tick pos=left,
x grid style={white!70!black},
xmajorgrids,
xlabel={Density of states (1/eV)},
minor y tick num=1,
grid=both,
y grid style={white!70!black},
ylabel={},
ymin=\YMINBSGW,
ymax=\YMAXBSGW,
ymajorgrids,
yticklabels={},
ytick style={draw=none},
legend pos=outer north east,
legend cell align={left},
]
\PLOTSDOSGW
\end{axis}
\end{tikzpicture}

\caption{Electronic band structure computed from the self-consistent field (SCF) electronic structure method (left) and computed from $G_0W_0$ (right). The energy of the bands is given relative to the valence band maximum (VBM).}
\end{figure}

\begin{figure}[h!]
\centering
\setlength\figureheight{11cm} 
\setlength\figurewidth{0.47\textwidth}
\begin{tikzpicture}
\begin{axis}[
height=\figureheight,
width=\figurewidth,
tick align=outside,
tick pos=left,
x grid style={white!70!black},
xmajorgrids,
xmin=0, 
xmax=\XMAXBSSCFSOC,
xtick style={color=black},
xtick={\TICKSSPECIALKPOINTSBSSCFSOC},
\XTICKLABELSSCFSOC
minor y tick num=1,
grid=both,
y grid style={white!70!black},
ylabel={$E-E_\text{VBM}$ in eV},
ymajorgrids,
ymin=\YMINBSSCFSOC,
ymax=\YMAXBSSCFSOC,
ytick style={color=black},
legend pos=north west,
]
\addlegendimage{empty legend}
\addlegendentry{SCF with SOC}
\PLOTSBSSCFSOC
\end{axis}
\end{tikzpicture}

\hfill
\begin{tikzpicture}
\begin{axis}[
height=\figureheight,
width=\figurewidth,
tick align=outside,
tick pos=left,
x grid style={white!70!black},
xmajorgrids,
xmin=0, 
xmax=\XMAXBSGWSOC,
xtick style={color=black},
xtick={\TICKSSPECIALKPOINTSBSGWSOC},
\XTICKLABELSGWSOC
minor y tick num=1,
grid=both,
y grid style={white!70!black},
ylabel={$E-E_\text{VBM}$ in eV},
ymajorgrids,
ymin=\YMINBSGWSOC,
ymax=\YMAXBSGWSOC,
ytick style={color=black},
legend pos=north west,
]
\addlegendimage{empty legend}
\addlegendentry{$G_0W_0$+SOC}
\PLOTSBSGWSOC
\end{axis}
\end{tikzpicture}

\caption{Electronic band structure computed from the self-consistent field (SCF) electronic structure method with spin-orbit coupling (SOC). The energy of the bands is given relative to the valence band maximum (VBM).
%
Blue bands have an odd band index (1, 3, 5, $\ldots$) and orange bands have an even band index (2, 4, 6, $\ldots$). 
%
The blue/orange color does not include information on the spin expectation value of the bands. 
}
    \label{f3}
\end{figure}




\section{How to reference the \GW calculation}
When using the \GW implementation in CP2K, please cite the following publications:
\begin{enumerate}[leftmargin=*]

\item[[1{]}] M.~Graml, K.~Zollner, D.~Hernangómez-Pérez, P.~E.~Faria Junior, and J.~Wilhelm, \textit{Low-scaling GW algorithm applied to twisted transition-metal dichalcogenide heterobilayers}, \href{
https://doi.org/10.48550/arXiv.2306.16066}{arXiv 2306.16066 (2023)}.

\item[[2{]}] J.~Wilhelm, P.~Seewald, D.~Golze, \textit{Low-Scaling GW with Benchmark Accuracy and Application to Phosphorene Nanosheets}, \href{https://doi.org/10.1021/acs.jctc.0c01282}{
J.~Chem.~Theory Comput.~\textbf{17}, 1662-1677 (2021)}.


\item[[3{]}] T.~D.~Kühne, M.~Iannuzzi, M.~Del\;Ben, V.~V.~Rybkin, P.~Seewald, F.~Stein, T.~Laino, R.~Z. Khaliullin, O.~Schütt, F.~Schiffmann, D.~Golze, J.~Wilhelm, S.~Chulkov, M.~H.~Bani-Hashe\-mian, V.~Weber,  U.~Bor\v{s}t\-nik, M.~Taillefumier, A.~S.~Jakobovits, A.~Lazzaro, H.~Pabst,  T.~Müller,  R.~Schade, M.~Guidon, S.~Ander\-matt, N.~Holmberg, G.~K.~Schenter, A.~Hehn, A.~Bussy,  F.~Belleflamme, G.~Tabacchi, A.~Glöß, M.~Lass, I.~Bethune, C.~J.~Mundy, C.~Plessl, M.~Watkins, J.~VandeVondele, M.~Krack, J.~Hutter, \textit{CP2K: An electronic structure and molecular dynamics software package - Quickstep: Efficient and accurate electronic structure calculations}, \href{https://doi.org/10.1063/5.0007045}{J.~Chem.~Phys.~\textbf{152}, 194103 (2020)}.


\end{enumerate}

\end{document}
